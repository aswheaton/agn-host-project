\documentclass[a4paper,11pt]{article}
\usepackage[utf8]{inputenc}
\usepackage[english]{babel} % Adds processing for some simple special characters.

% Packages for better page formatting and nice footnotes. %

\usepackage{fancyhdr} % Headers.
\usepackage{multicol} % Columns.

\renewcommand{\thefootnote}{\fnsymbol{footnote}}

% Utilities for fine-grained control over image insertion. %

\usepackage{graphicx} % Insert images.
\usepackage{float} % Images float in document environment.
\usepackage{wrapfig} % Image/tables in multicols with \wrapfigure or \wraptable.
\usepackage[justification=centering]{caption} % Captions for single images.
\usepackage{subcaption} % Captions for simultaneous images.
\graphicspath{{./img/}}

% Some utilities for scientific and mathematical writing. %

\usepackage{siunitx} % Formatting for numbers with SI units.
\usepackage{amsmath} % Astronomical symbols.
\usepackage{isotope} % Nice markup syntax for chemical symbols.
\usepackage{xfrac}   % Slant fractions and other utilities.
\usepackage{lipsum}  % Generate pseudo-text for formatting.
\usepackage{listings}

% Utilities for manual kerning adjustment. %

\newcommand\K{\kern.05em} % Small amount of kerning.
\newcommand\KK{\kern.1em} % Medium amount of kerning.
\newcommand\KKK{\kern.2em} % Large amount of kerning.
\newcommand\KKKK{\kern.3em} % Very large amount of kerning.

\begin{document}

\noindent \newline \textbf{Welcome}

\noindent \newline Hello! and thank you for coming to my talk, today. My name is Alex Wheaton,
and I'm going to talk about AGNs, tidal disruption events, star formation,
and the relationships between all three.

\noindent \newline The aim of my project was to investigate the statistical behaviour of the star formation in the host galaxies of tidal disruption events.

\noindent \newline \textbf{Motivation}

\noindent \newline I was given access to the spectra and photometry for eight galaxies which
hosted tidal disruption events within the past few years. There's a small
table of them here, and as you can see, they are all similarly of low
magnitude and moderate redshift. Because they are all at redshifts <0.1, we
can say that they are approximately the same age as our own Milky Way.

\noindent \newline At these distances, they occupy very little real estate on a CCD, but the
XSHOOTER instrument on the Very Large Telescope provides us with very high
resolution spectra from these objects.

\noindent \newline \textbf{Active Galactic Nuclei (AGN)}

\noindent \newline Now, if you're in the physics department you are probably familiar with the fact
that most galaxies are host to a supermassive black hole at their centres. The
vast majority of these, including our own Milky Way, emit little to no light which we can
detect, and are said to be ``quiescent'', but a small percentage of galaxies
have a core luminosity which is a significant fraction of the luminosity of the
whole.

\noindent \newline \textbf{Active Galactic Nuclei (AGN)}

\noindent \newline The spectral energy distribution of this luminosity is broad featuring peaks in
the X-ray, UV, and IR with strong and very broad emission lines. This type of
spectrum is not explicable solely by blackbody emission, and so is believed to
be the emission from matter accreting onto the surface of a black hole, at a
variety of temperatures and orbital velocites.

\noindent \newline \textbf{The Starburst-AGN Connection}

\noindent \newline Now, for decades, a correlation between nuclear activity in distant galaxies
star formation in those galaxies has been emerging. Many AGN have host galaxies
with atypically blue spectra, indication recent star forming activity (more on
that in a moment). And as you can see from this figure, there is a tight relationship
between total stellar luminosity and black hole mass. In other words, and positive
correlation between star formation and black hole accretion.

\noindent \newline The recent component of star formation is referred to as a "starburst"---a
short period of abnormally high star formation which occurs long after most of
the stellar mass in a galaxy is formed.

\noindent \newline \textbf{Possible mechanisms?}

\noindent \newline There are a couple of proposed mechanisms for this action. It's possible that
radiation from AGN activity heats interstellar gas such that it can not collapse
gravitationally to form new star. It also might be that stellar wind shuts of
the supply of fuel to AGN. But since these processes take place over the course
of billions of years, we can only see individual galaxies in a snapshot. They
are either active or not, they are star forming---or they are not. We can't
observe the dynamics of this process.

\noindent \newline \textbf{Tidal Disruption Events}

\noindent \newline Now, another type of nuclear activity is a tidal disruption event, or TDE for short.
This occurs when a single star makes a close orbital approach of an otherwise
quiescent black hole. Tidal forces on the star rip it apart. Some material falls
into the black hole and the rest is jettisoned on a wide orbit. This process
produces a brief but intense increase the nuclear luminosity, which then fades
over a matter of days to months. In contrast to typical AGN activity and star
formation, this type of event is a transient---we can watch it happen in real.

\noindent \newline But how are tidal disruption events related to star formation? In this project,
I've used the XSHOOTER data on TDEs from the Very Large Telescope to investigate
whether or not the host galaxies of transient tidal disruption events exhibit
the same statistical behaviour as those of always-on AGN.

\noindent \newline \textbf{The BAGPIPEs Module}

\noindent \newline To do this, I employed the BAGPIPEs Python module, authored in 2019 by one of
Royal Observatory's very own, to infer the star formation histories of the TDE
hosts from their spectra.

\noindent \newline \textbf{The BAGPIPEs Module}

\noindent \newline BAGPIPEs has two main functions. First, it allows you to define the functional
form of a simulated galaxies star formation history, in stellar mass formed per
unit time, over the history of the universe. From that prior, it can simulate
that galaxy's spectrum as it would appear to us in the present day.

\noindent \newline \textbf{The BAGPIPEs Module}

\noindent \newline It can also do the inverse: take in a real, observed spectrum from a galaxy and
infer from this a distribution of posterior star formation histories.

\noindent \newline \textbf{Stellar Population Dating}

\noindent \newline It does this using stellar population dating, whereby the ratio of flux from
massive blue stars is compared to that from less massive red stars to make assumptions
about when the stellar population formed.

\noindent \newline The reason this works is because very massive, blue stars live much shorter than
their smaller counterparts, so galaxies with a high number density of blue stars
necessarily must have had recent star formation. The problem with this, as you
can see from this table, is that stellar lifetimes increase dramatically with
decreases in initial mass.

\noindent \newline \textbf{Stellar Population Dating}

\noindent \newline The number O and B type stars in a given galaxy can give you detailed
information about the past billion years. But while the next smallest A type
stars are on the main sequence, it's difficult to determine whether they were
formed a billion years ago or three billion. As you go down the spectral classification
types, the problem only gets worse. As this table illustrates, the universe is
mostly populated by main sequence red dwarfs, which live for many many billions
of years.

\noindent \newline \textbf{Stellar Population Dating}

\noindent \newline So before I applied BAGPIPES to the TDE data, wanted to do some tests to see
what it was really capable of detecting. I initially tried to look for signatures
of star formation than just the ratio of blue to red light in the host galaxies.

\noindent \newline \textbf{}

\noindent \newline So I simulated a galaxy with a simple, Gaussian star formation component, at
many different points in time, from a distant redshift 2.2 to the near present
z=0.01.

\noindent \newline This is the variance in flux at various wavelength as the star formation is
evolved towards the present. Although I had hoped that certain emission lines
would indicate past star formation, I actually found that most metal lines vary
LESS than the overall spectrum, which is mostly affected in the blue region.

\noindent \newline \textbf{}

\noindent \newline Looking at deltas of these flux values rather than the variance, unfortunately
yields similar results.

\noindent \newline \textbf{SFH Inference - What is possible?}

\noindent \newline I next decided to do some blind testing of BAGPIPES fitting ability---wherein
I simulated a spectrum with a known SFH, then piped that spectrum back through
the fitter to see if I could reconstruct the SFH from the spectrum, without any
prior knowledge of it.


\noindent \newline \textbf{R1: Entire Parameter Space with Two Components}

\noindent \newline I failed several times. My first attempt, which I'll call R1 here, allowed the
fitting routine to explore almost the entire parameter space of star formation
history.

\noindent \newline This is a 13-dimensional fit, that is there are 13 parameters simulated---from
the mass and age of the different components, to their shape, metallicity, and
other galactic properties like velocity dispersion and nebular emission.

\noindent \newline Fitting a two component model with both a bulk component and burst
that are allowed to vary over the whole history of the host usually misses the
burst component, instead favouring a younger bulk component in the posterior
than in the prior, to account for the blue stars produced in the burst.

\noindent \newline \textbf{R2 \& R3: Iterative Fitting}

\noindent \newline To make my fit a little more physically realistic, I insisted that the burst
component be younger than the bulk component. I did this by first fitting a one
component bulk formation to the spectrum, and then testing to see whether
refitting with the addition of a younger burst improved the fit.

\noindent \newline \textbf{R2 \& R3: Iterative Fitting}

\noindent \newline Here you can see the fit (on the bottom) misses the burst in the prior (on top).
Instead, it favours a younger, single component of bulk formation, and that
adding a burst component doesn't shift the older one to the age in that of the prior.

\noindent \newline \textbf{Selecting Physically Reasonable Priors}

\noindent \newline So I went back to the drawing board. Clearly, the fit needed narrower constraints.
I went to the literature, and found that as we peer into the cosmos at high
redshift galaxies, star formation actually peaks at a time called "cosmic noon",
around redshift 3-4.

\noindent \newline If I insisted that this be the case in my posterior star formation histories,
perhaps I could avoid the erasure of the burst component in the fit.


\noindent \newline \textbf{R4: Fixed Old Component, Free Burst Component}

\noindent \newline With that in mind, I devised a new set of priors, insisting that most of the
stellar mass be formed in an earlier era, and letting the burst float freely
over the parameter space. If a nonzero burst mass was detected, I could
reliably infer the existence of young, high mass stars in the host. If the burst
mass was found to be near zero, then the stellar population must be old indeed.

\noindent \newline \textbf{Selecting Physically Reasonable Priors}
\noindent \newline \textbf{R4: Fixed Old Component, Free Burst Component}

\noindent \newline At this time, I also refined some other parameters be be consistent with
literature values for my the redshifts and Hubble Morphologies of my TDE hosts.

\noindent \newline \textbf{Blind Fitting Results with R4 Priors}

\noindent \newline Finally, I began to successfully detect the starbursts. Here you can see one
such "blind fit", again with the "true" SFH on top, and the fitted one on the
bottom.

\noindent \newline \textbf{Blind Fitting Results with R4 Priors}

\noindent \newline When a burst of sufficient mass \textit{is} present in the prior, it detects
them, although sometimes it struggles to pin down the age of bursts older than
3 billion years.

\noindent \newline \textbf{Blind Fitting Results with R4 Priors}

\noindent \newline Importantly, fitting with these priors doesn't give false positives. When there
is no burst in the prior, it doesn't show up in the fit either.

\noindent \newline \textbf{Blind Fitting Results with R4 Priors}

\noindent \newline Very low mass bursts, such as this one, are sometimes missed.

\noindent \newline \textbf{Blind Fitting Results with R4 Priors}

\noindent \newline Bursts of moderate mass (around 10 to the 8 solar masses) are detected, but their mass
sometimes underestimated, as you can see here. I thought this might be due to the
fit favouring longer decay times for older component, so that some blue stars
are accounted for by ongoing formation instead of the burst.

\noindent \newline \textbf{Correlated Parameters}

\noindent \newline Upon investigation though, I found that the old component decay time and burst
mass were not very strongly correlated, so the reason for mass underestimation
still merits some study.

\noindent \newline \textbf{Application to XSHOOTER Data}

\noindent \newline Now confident in the reliability of fits performed with these priors, I turned
BAGPIPEs on the TDE data, which of course has no "known" SFH. Here are some of
my results:

\noindent \newline \textbf{Application to XSHOOTER Data}

\noindent \newline As you can see, the TDE hosts exhibit strong evidence of recent starburst activity,
with star formation rates much higher than the typical ambient rate of 2-3 solar
masses per year.

\noindent \newline \textbf{Application to XSHOOTER Data}

\noindent \newline Almost all of the fits find old components which are consistent with observational
evidence for star formation at cosmic noon, and with burst components consistent
with present day starburst galaxies.

\noindent \newline \textbf{Application to XSHOOTER Data}

\noindent \newline \textbf{Application to XSHOOTER Data}

\noindent \newline \textbf{Application to XSHOOTER Data}

\noindent \newline \textbf{Conclusions}

\noindent \newline In summary, it seems that starburst activity in the past 3 billion years of a
galaxy's history is inferable with some reliability, given that certain assumptions
about what constitutes a physically realistic SFH are made. With that in mind, I've
applied this to the spectra of several TDE hosts, and found that they indeed
seem to share the statistical behaviour of in-transient AGN.

\noindent \newline \textbf{Future Applications}

\noindent \newline With that said, it's important to note that eight samples available to me for
this project are not nearly a large enough sample size to draw this conclusion.
This "proof of concept" has applications for larger research projects, and it
would certainly be valuable to apply fits with the R4 priors to larger sample
size, with priors more carefully tailored to host morphology, and to compare
these results to quiescent galaxies with similar morphologies.

\noindent \newline With those questions, I'll conclude my talk! Thank you for coming, and questions?

\end{document}
